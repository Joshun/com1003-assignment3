\documentclass[a4paper]{report}
\usepackage{titling}
\usepackage{lipsum}
\usepackage{parskip}

\newcommand{\sign}[1]{%   
  \begin{tabular}[t]{@{}l@{}}
  \makebox[2in]{\dotfill}\\
  \strut#1\strut
  \end{tabular}%
}
\newcommand{\Date}{%
  \begin{tabular}[t]{@{}p{2in}@{}}
  \\[-2ex]
  \strut Date: \dotfill\strut
  \end{tabular}%
}

\newcommand{\subtitle}[1]{%
  \posttitle{%
    \par\end{center}
    \begin{center}\large#1\end{center}
    \vskip0.5em}%
}

\begin{document}

\title{Lego Robot Assignment}
\subtitle{COM1003}
\date{\today}
\author{Jack Deadman\\ Joshua O'Leary}
	
\maketitle

\section*{Statement of Contribution}

The majority of the code was produced by pair programming, however we each focussed on some areas more than others:

\begin{minipage}{8cm}

\end{minipage}

I focussed on algorithm for turning at corners (when detecting an obstacle), in addition to improving the overall structure of the code.
\begin{figure}[h]
	\begin{minipage}[c]{0.4\linewidth}
    	\sign{Mr Jack Deadman}
    	\Date
	\end{minipage}

\end{figure}

I focussed on the functions directing the robot's movement, in addition to the functions for obstacle and black detection.
\begin{figure}[h]
	\begin{minipage}[c]{0.4\linewidth}
		\sign{Mr Joshua O'Leary}
		\Date
	\end{minipage}	
\end{figure}

\end{document}